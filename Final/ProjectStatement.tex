%\section{Project Statement: Utilizing unused renewable Energy}
\begin{parcolumns}[colwidths={1=2.5 cm, 2=10 cm, 3=2.5cm}]{3}

\colchunk{  \\ \\ \textbf{Current Situation} \\ \\ \\ 
				\\ \\ \\	\\ \textbf{Need}\\ \\
				\\ \\ \\ \\ \\	\\ \textbf{Problem}\\ \\ \\
				\\\\\\\\\\\\\\\\\\\\\\ \textbf{Method and Criteria}
				\\ \\ \\\\\\\\\\\     \\\\\\ \\\\\\\\\\\\\\\\\\	\\\\
				\textbf{Aspects covered} \\\\\\\\\\\\\\\\\\\\\\\\\\\\ \textbf{References and Informations}}

\colchunk{\\ \\Each year the electricity generated by using renewable energies like solar and wind makes up a bigger part in the energy-mix \protect\footnotemark of Germany. But the amount varies because of seasonal or just daily fluctuations. This is especially true for wind. In 2018 wind energy was the main renewable source with about 48\%, which  made up around 19\% of the overall consumption.\\ \\
Because of the significant mismatch in grid power demand, the need for a solution is becoming more acute. It’s a well-established problem for the industry, and there are a number of energy management and storage systems in the pipeline today, which could solve this problem. But few offer a complete solution allowing wind energy to be seamlessly plugged into the grid.\\\\
Today, the importance of transitioning into a sustainable and cost-effective energy sector is more important than ever. Fossil fuels won’t last for ever and are straining the environment too much. The state will on the long-term ban or at least heavily restrict the usage to meet its own agendas therefore the solution for efficiently using renewables is of utmost importance.\\
Energy production by wind power is intermittent and fluctuates. Currently one of the main challenges is the adaptability of different energy storage or management systems to daily and annual fluctuations as well. This paper will investigate which is the best solution to those problems.\\ \\
The new solution must be more cost-efficient than just shutting the wind turbines off, or buying energy from other countries. It has to be able to be integrated into the current grid of wind turbines. The factors to rate our solution therefore include:
\begin{itemize}
\item \textbf{Costs} Must be equal or lower than 21.82 billion \euro over 10 years: including investment, maintenance and operating costs. Otherwise, it is cheaper to than just shut the wind turbines off, or buy energy from other countries.
\item \textbf{Efficiency} Must be equal or higher than70 $\%$: including kwh lost while transforming and lost while saving over one year.
\item \textbf{Safety} in $\%$: failure rate per year must be lower than 1 ppm (part per million)
\item \textbf{Scaling} yes or no: Is it reasonable for an input of 1,495 GW per hour and saving 36 GWh? 
\item \textbf{Technical Feasibility} yes or no: Implementable in the next 5 years? Is the technology viable or is something better obtainable in the next years? Is it possible in the geographic area?  
\\
\end{itemize}
The most promising Solutions are covered in this analysis:
\begin{itemize}
\item \textbf{Storage systems made out of batteries} \newline Written by CFO Annabelle. See section IV page ???
\item \textbf{Storage systems made out of Pump storage} \newline Written b COO Lennart. See section IV page \pageref{PumpedStorageHydropower_Lennart}
\item \textbf{Power to Gas} \newline Written by CAO Christian. See Section IV page \pageref{PowertoGas_Christian}
\item \textbf{Storage systems made out of hydrogen} \newline Written by CEO Moritz. See section IV page \pageref{H2_Moritz}
\item \textbf{Storage systems made out electric vehicles}\newline Wtitten by CTO Kai. See sectition IV page \pageref{V2G_Kai}
\\
\end{itemize}
[energy-mix]  Der deutsche Strommix: Stromerzeugung in Deutschland. \\https://strom-report.de/strom/.\\Last accessed 21.11.19


}

\colchunk{\\\,\,\,\,\, \textbf{References}
\\\\\\ 1 Source: \url{https://strom-report.de/strom}}

\end{parcolumns}