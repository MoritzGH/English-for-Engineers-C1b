\subsection{Batteries}
\cfoot{Annabelle}
\begin{parcolumns}[colwidths={1=2.5 cm, 2=10 cm, 3=2.5cm}]{3}
\colchunk{\underline{\textbf{Intro}}\\ \\\\ \\ \\ \\ \\ \\ \\ \\ \\ \\ \\ \\ \\ \\ \\ \\ \\ \\ \\ \\\\ \\ \\ \\ \\ \\ \\ \\  \underline{\textbf{Findings}} \,\,\textbf{\,\,\,\,\,\,\,\,\,\,\, \,\,Types of \, \, \, \, \, Batteries and their advantages} \\  \textbf{Lithium-ion \, battery}\\ \\
				\\ \\ \\ \\ \\ \\ \\ \\ \\   \textbf{Nickel-Cad-\,\,\,\, \, \,mium battery}\\ \\ \\
				\\ \\ \\ \\ \\	\textbf{Sodium-Sulfur battery}
			\\ \\ \\ \\ \\ \\ \\ 	\\ \\ \\ \\	\textbf{Flow battery}
			\\\\\\\\\\\\	\\ \\ \\ \\ \\ \\ \\ \\ \\ \\ \\ \\ \\ \\ \\ \\ \\ \\ \\ \textbf{Electro- \,\,\, \, \, \, \,chemical \, \, \, \, \, Capacitor}
			\\\\\\\\\\\\ \\ \\\\ \\ \\ \underline{\textbf{Problems}}\\ \underline{\textbf{encountered}} \\ \\ \\ \\ \\ \\ \\ \\ \\ \\ \\ \\ \\\\ \\\ \\ \\ \\ \ \\ \\ \\ \\ \\ \\ \underline{\textbf{Current}}\\ \underline{\textbf{Developments}}\\\\ \\ \\ \\ \\  \underline{\textbf{Solution}}\\ \underline{\textbf{Discussion}}\\ \textbf{Costs}
	\\	Determining the Budget \\ \\ \\\\ \\ \\ \\ \\ \\ \\ \\ \\ \\ \\ \\ \\Investment costs \\ \\ \\\\ \\ \\ \\ \\\ \\ \\\\\  Operating costs \\\\ \\ \\ \\ \\  \textbf{Efficiency} \\ \\ \\ \\ \\ \\ \\ \\ \\ \textbf{Safety} \\ \\\\ \\  \\\\ \textbf{Scaling} \\\\ \\ \\  \\ \\ \textbf{Technical Feasibility} \\ \\ \\ \\ \\ \\ \textbf{VRFB vs NaS battery}\\ \\ \\ \\ \\ \\ \\ \\ \\ \\   \textbf{Sustainability} \\ \\\\ \\  \underline{\textbf{Conclusion}} \\ \\ \\ \\\ \\ \\\\ \\\underline{\textbf{Recommen-}} \\\underline{\textbf{dation}} \\ \\ \\ \\\\\\ \textbf{Personal Comments}
			}
			
\colchunk{Today's electricity is more and more produced by renewable sources such as wind turbines. Because of the intermittent power supply, due to fluctuating winds, there is a need for storing surplus electricity. In some cases wind turbines need to be taken of the grid to avoid an overload. Additionally Germany pays other countries to take its surplus electricity while still paying 1.5 billion \euro\, per year for importing energy in case of electricity shortage. 
\\ A major portion of today's energy storage systems for portable electronic devices, such as mobile phones or electric vehicles, are rechargeable batteries. They contain an anode, a cathode and different kinds of electrolytes, which affect their specific field of application. \\
This research will concentrate on 5 types of  batteries to determine which kind is most suitable for storing wind energy under the given requirements. Starting with costs, which is the most important criteria, because it determinates if the storage system can be built at all. In a time span of 10 years the investment, maintenance and operating costs cannot exceed 21.82 billion \euro for a capacity of 36\,GWh. Next is efficiency, the aim is to retain at least 70\,$\%$ of the initial energy. Another aspect to energy storage, which needs to be fulfilled, is the capability to adapt the system for single wind turbines or a whole wind park. This should be achievable while also being safe and implementable in the near future. 
Regarding Batteries there are a few particularities concerning their limit in energy density, ability to hold electricity over a long period of time and life cycle durability.  \\
Lithium-ion battery (short LIB) is the umbrella term for batteries which transfer an intercalated lithium compound through an electrolyte between the electrodes during the discharge and charge process. Typically the anode (negative electrode) consists of lithium titanate or graphite and the cathode (positive electrode) consists of lithiated metal oxides or phosphates. Compositions with a higher cell voltage (around 3.6\,V to 3.7\,V) and high energy density are less safe than cells based on lithium iron phosphate or lithiated metal oxide. Therefore the energy density and the voltage (between 3.2\,V and 2.5\,V) is a lot lower. LIB have a low self-discharge and don't suffer from a memory effect. Compared to newer technologies such as sodium sulfur batteries, they have shorter cycle life and for that reason are less profitable in a large scale energy storage.\\ Nickel-cadmium batteries (NiCd batteries) have a nickel positive and a cadmium negative. They excel at maintaining voltage (around 1.2\,V)  and delivering their full rated capacity at high discharge rates. Some of their disadvantages are the cost-intensive materials, which they are made out off and their high self-discharge rates. Because NiCd batteries are best stored discharged, they are not suited as an energy storage system, even though they over a lot of advantages.\\A sodium-sulfur battery (NaS battery) is a molten-salt battery. The anode (sodium) and cathode (sulfur) are liquid and the electrolyte membrane is a solid ceramic (sodium alumina), which also separates the electrodes. Two features are the high operating temperature between 300 and 350$^\circ$C  and the fast corrosion of the sodium polysulfides. Therefore, the NaS batteries are primarily used as stationary energy storage applications. They offer a lot of advantages, such as high energy density, high efficiency (around 89$\%$), long cycle life and affordable materials for the construction. The cell voltage amounts to 2\,V.
\\A flow battery, more precisely a reduction-oxidation flow battery (redox flow battery), mainly consists of two liquid electrolyte solutions which flow through electrochemical cells and are separated by a ion-selective membrane. There are various types of flow batteries such as hybrid redox flow batteries or redox batteries without a membrane. Compared to a conventional battery the difference is that energy is stored in the electrolyte rather than in the electrode material. The cell voltage is specific to the chemical components involved in the reactions and the number of cells that are connected in series. In practical applications it reaches 1 to 2.2\,V. 
The amount of energy stored is depending on the amount of active chemical species present in the electrolyte. Therefore the capacity can be easily amplified by expanding the size of the storage tanks. Because the amount of electrolyte flowing in the electrochemical cells is usually only a small fraction of the total amount, uncontrolled energy release only concerns this small portion. Another advantage are the unrivalled cycle life (compared to other batteries) of 15,000 to 20,000 cycles for vanadium redox flow batteries (VRFB), which are the most commonly used redox flow batteries. 
The main issue up to this date is that flow batteries are less powerful and have a more lavish structure. \\
An electrochemical capacitor (EC), also known as supercapacitor or ultracapacitor, stores energy using ion adsorption or surface redox reactions. For storing energy over night, meaning 1 cycle per day, an asymmetrical EC with a lead oxide positive electrode and an activated carbon negative electrode is most suitable. Even though its energy density (around 20 Whkg$^{-1}$) is lower compared to batteries and its best adapted for a short duration of storing energy, it offers a lot of different advantages. Such as a 10 to 100 times higher capacity (energy per unit volume or mass), fast acceptance and delivery of power, good reliability and unmatched cycle life (over 500,000).  \\
Trying to decide which battery or capacitor is the most suitable, is not an easy task. Nickel-cadmium batteries are eliminated from the selection quickly because they have a high self discharge rate and cannot keep up with newer technology in this instance. Deciding whether or not LIBs are a viable option is not that easy. There is a lot more research and practical applications for LIBs on which one can build upon. Still, they are not a long term solution. It's not the aim to find a storage system which needs to be renewed every  3 years, but to find a durable and adaptable one. \\ Another problem is that the operating costs are unknown. For most sources it's not clearly stated if the operating cost is included in the price per kWh. Estimating the costs is quit complicated, thus the batteries are ranked from least to most cost-intensive. The EC ages slowly and has a temperature range from -40 to +70\,$^\circ$C, so it's not high-maintenance. Vanadium redox batteries do not need to be cooled because they have a integrated cooling system due to the flow of electrolytes which dissipates heat. Therefore they don't need a fire protection system either. Most maintenance is needed for the sodium-sulfur battery. It's prone to corrosion of the insulators and growth of dendrites, which increases self-discharge.\\  
Another issue is, that there's still a lot of research done compared to more established battery types such as nickel-cadmium or lithium-ion batteries. Especially ECs are in development and can be further optimized. In a few years time those technologies will be more affordable and sophisticated, as the demand will increase. Nevertheless an investment at this point is reasonable. 
\\
The expenses for covering Germany's demand of electricity in case of an electricity shortage amounts to around 1.5 billion\,\euro\,\,in 2017. In this year Germany exported 83.4 billion kWh, imported 28.4 billion kWh and produced 105.5 billion kWh of energy with wind turbines. Consequentially there is an excess of electricity, which can be stored. Because there are no specific values for how much energy is needed at a time, the minimum capacity can be deduced from the quantity of lost power per month. According to the BDEW (Federal association of energy and water economy) approximately 3.23 billion kWh could not be fed into the grid from January to March of 2019. So on average this amounts to 1.495 GW of nominal capacity and a storage capacity of 36 GWh (one day of full nominal capacity). Adding the price per kWh for the additional 12,92 TWh gives a budget of 2.18 billion\,\euro\, per year and 21.82 billion\,\euro\, for 10 years. Overall this adds up to 131 TWh over 10 years, presuming that Germany will still be producing as much electricity in 10 years as now.  \\
Taking the values from table 1 (475\,\euro\, for NaS, 330\,\euro\, for VRFB and 18,000\,\euro\, for EC), the most affordable option are vanadium redox flow batteries with investment costs of 11.88 billion \euro. Next up are sodium-sulfur batteries with 17.1 billion \euro\, and lastly electrochemical capacitors with 648 billion\,\euro. The costs for the ECs might be unrealistic compared to the other options, because one needs to consider their high cycle life, which makes them more lucrative over a longer period. 10 years are only 3650 cycles, assuming that batteries discharge over the day and charge over night when wind increases. \\
Because corrosion of insulators and growth of dendrites, which increases self-discharge, are an issue with sodium-sulfur batteries, operating costs are potentially higher. VRFB do not need to be cooled because of the flow of electrolytes which dissipates heat, nor do they need a fire protection system. \\
Both batteries are rather efficient (NaS 85$\%$ and VRFB around 75$\%$), so they fulfil the criteria.
NaS batteries operating temperature is around 290\,$^\circ$C, but they do not need to be heated, because one cycle per day heats the battery enough due to ohmic loss. This leads to 2$\%$ loss per 300 cycles self-discharge rate. For the vanadium redox flow battery no self-discharge rate could be found, but it's possible that the 75$\%$ efficiency includes this aspect.\\
In the safety criteria the VRFB is leading, because it is non-flammable and inherently safe. There is no accident known, unlike with NaS batteries. One accident happened 2011 in the Tsukuba plant, leading to a fire, but a statistic stating what percentage the risks are, is not public. \\
Again the VRFB offers a distinct advantage compared to the sodium-sulfur battery. Power and energy are completely separated and can be scaled up or down and are highly flexible. If the demand for storage options expands, VRFB can easily be upgraded by adding bigger electrolyte tanks.\\
It was not possible to find data stating how much time it would consume to build either of the battery types. Most definitely the construction of a VRFB would cost more time because of its sophisticated structure. Altogether the implementation of both batteries should not take more than 5 years considering that there are some companies in Europe producing them.\\
VRFBs reaction time to frequency changes is rather slow. The frequency can be restored after a over-stressing of the grid with its help. Sodium-sulfur batteries have such a quick reaction time (1\,ms), that they can prevent any frequency irregularities altogether. This makes both of them helpful additions to the grid. Concerning the charging and discharging performance the vanadium redox flow battery is superior to a NaS battery. It can discharge up to 20 hours while NaS batteries only last about 6 hours. Additionally VRFB are able to replenish charge and provide power simultaneously.\\ 
The options at hand might not be the most sustainable options due to their use of rather rare metals, but sodium sulfur is a non toxic battery and vanadium can be recycled.\\
Sodium-sulfur batteries and vanadium redox flow batteries both offer great advantages, which makes deciding which one is the most suitable complicated. Considering that the demand for energy might expand greatly, the VRFB is the best decision for the future. A storage system for 36 GWh costs 11.88 billion \euro and will last approximately  40 years. Operating costs still have to be applied but should be moderate.\\
It is recommended to start construction of a vanadium redox flow battery as soon as possibly. Due to the shut-down of nuclear power plants, renewable electricity will become  a major source of electricity in Germany. To be energy-self-sufficient must be an objective. The best implementation is one big storage facility near a wind park which has enough space for later expansions. \\
It was not easy to find consistent data concerning the price per kWh for any of the battery types. The spectrum reached from 0.05\,\$ per kWh to 20,000\,\$ per kWh for electrochemical capacitors. Some of those values refer to a cost projection after an unspecified amount of years, making it incomparable. The reason a lot of the sources are wikipedia sources is that a lot of articles cost money and I was not up to pay 35\,\$ per PDF. So I learned not to double check every value I found with another source, but to accept insufficient research.


}

\colchunk{\\\,\,\,\,\,\\ \\ \\ \\ \\ \\ \\ \\\\\\\\ \\ \\ \\ \\ \\ energystorage.org
\\\\\\\\\\\ \\ \\ \\ \\ \\ \\ \\ \\ \\ \\ \\ circuitdigest.com \\ \\ \\ \\  \\ \\ \\\\\\
wikipedia.org
\\ \\ \\ \\ \\ \\ \\ \\ \\ \\ \\ \\  wikipedia.org \\
energystorage.org
\\ \\ \\ \\ \\ \\ \\ \\ \\ \\ \\ \\ \\ \\ \\ \\ \\ \\ \\ \\ \\ \\ \\ \\  wikipedia.org\\nature.com \\ John R. Miller and Andrew F. Burke (2008)\\
\\ \\ \\ \\\\ \\  \\ \\  Federal Network Agency and Federal Cartel Office \\energiefirmen.de
}

\end{parcolumns}
\begin{table}[H]
\centering
\caption{Specific values of different battery types}
\begin{tabular}{cccc}
\toprule
& NaS battery & VRFB & EC\\
\midrule
Cycle life (90\% capacity drop) & 15 years and 4.5k&12k-20k& 500k\\
Investment costs [\$ per kWh] & 50-525 & 364.44-1078&0.05-20,000\\
Investment costs [\euro \,per kWh]& 45-475 & 330-970& 0.04-18,000\\
Operating costs& high & middle & low\\
Efficiency of conversion & 85\% DC & 65-80\% (commonly 75\%) & 75-95\%\\
Specific energy [Wh per kg] & 150-240 & 10-20&0.5-15\\
Energy density [Wh per L] & & 15-25& 50\\
Operating temperature [$^\circ$]& 270-360& -5-50& up to 65 \\ 
Speed of charge/discharging &6\,h discharge &20\,h&5\,h\\
Features& non toxic & need a lot of space & \\
&start up speed of 1\,ms& non flammable, inherently safe &\\
\bottomrule
\end{tabular}
\end{table}
\textbf{Sources} \\batteryuniversity.com\\
solarbay.com.au\\
wikipedia.org\\
Journal of Advanced Chemical Engineering: Mark Moore... 2015\\
RWTH Aachen 2018\\
Haisheng Chen: Storing Energy 2016\\
Saurabh Tewari: Energy storage for smart grids 2015\\
\clearpage
\cfoot{}