\subsection{Vehicle to Grid}
\label{V2G_Kai}
\cfoot{Kai Braungardt}
\begin{parcolumns}[colwidths={1=2.5 cm, 2=10 cm, 3=2.5cm}]{3}

\colchunk{  \\ \\ \textbf{Introduction} \\  \\ \\ \\ \\ \\ \\ \\ \\ \\ Charging direction \\ \\ \\ \\ \\
				\\ \\ \\	\\ \\ \\ \\ \\ \\ \\  \textbf{Findings}\\ \\
				\\ \\ \\ \\ \\	 \\ \\ \\  \\ \\  \\ \\ \\ \\ \\ \\ \\ \\ \\ \\ Hypothetically scenario\\ \\ \\ \\ \\ \\ \\ \\ \\ \\ \\ \textbf{Solution Discussion}\\Scaling \\ \\
				\\\\\\\\\\\\\\\\\\\\\\\\\\\\\\\\\\\\\\\\\\\\\\\\\\\\\\ Cost
				\\ \\\\\\\\\\\\\\\\\\\\\\ \textit{Cost per station}\\\\\\\\\\\\\\\\\\\\\\\\\ \textit{Total costs}   \\\\\\\\\\	\\\\\\\\\\ Technical Feasibility \\ \\ \\ \\ \\ \\ \\ \\ \\ \\ \\ \\ \\ \\ \\ \\ \\ \\ \\
Effciency \\ \\ \\ \\ \\ \\ \\ \\ \\ \\ \\ \\ Substainability \\ \\ \\ \\ \\ \\ \\ \\ Safety \\ \\ \\ \\ \\ \textbf{Conclusion} \\ \\ \\ \\ \\ \\ \\ \\ \\ \textbf{Recommen -dations} \\ \\ \\ \\ \\ \\ \\ \\ \\ \\ \\ \\ \\ \textbf{Personal Comments}
}

\colchunk{\\ \\
One solution to utilize unused renewable energy, which does not rely on building additional storage systems is to use EVs (Electric vehicles) and PHEVs (Plug in Hybird) as a storage system.  When the power output of the grid/the offshore parks is low, the EVs can throttle their charging rate or even return power to the grid. The EVs could also delay their charge and use the peaks in the power output of offshore parks to charge their batteries.
%https://en.wikipedia.org/wiki/Vehicle-to-grid
\\\\
There are two fundamental ideas in Vehical to Grid. Bidirectional Vehical to Grid where the EVs also return power to the grid or unidirectional Vehical to Grid where the EVs only store the power but do not return power to the grid.\
Bidirectional Vehicle to Grid requires special hardware. This results in a system, that is far more complex and expensive than unidirectional Vehicle to Grid. It also results in a lot of additional wear in the EVs batteries. Therefore it would be a lot more difficult to convince customers to use bidirectional Vehicle to Grid. At the same time multiple studies have shown, that the profit is not significantly higher than with unidirectional Vehicle to Grid. Because of this unidirectional Vehicle to Grid is superior and will therefore be the object of the following calculations.
%\url{https://www.isi.fraunhofer.de/content/dam/isi/dokumente/sustainability-innovation/2010/WP4-2010_V2G-Valuation.pdf}
%\url{https://www.erneuerbar-mobil.de/sites/default/files/publications/anhang-optum-ap6_1.pdf}
\\ \\
\noindent
For the purpose of a large scale storage system for unused wind energy unidirectional Vehicle to Grid is not a viable option.
The necessary system should have a regulating power of 1495 MW and a capacity of 36 GWh is needed. This equals the average unused power in the first months of 2019. The capacity equals the amount of this power over 24 hours. Since the system is not only storing power but also constantly using it to drive the vehicles this should be enough capacity.
%https://www.bdew.de/presse/presseinformationen/zahl-der-woche-gut-32-mrd-kilowattstunden/
\\ \\
The unidirectional Vehicle to Grid system necessary to achieve this size would need over 10 million EVs or PHEVs. The problem is not the power. The necessary power output would only require a little more than 1.7 million EVs or PHEVs. But the storage to power ratio in Vehicle to Grid is a lot smaller than the necessary ratio. This results in a system with the necessary 36 GWh of storage capacity, but 9000 MW of regulating power. This system would need over 10 million vehicles. It is unlikely that this amount of EVs and PHEVs will be available in the next five years.
\\ \\
But even with 10 million EVs and PHEVs in the year 2025 the necessary infrastructure for a system this large would take a lot longer than five years to be built. In order to include 10 million vehicles in the system more than 10 million chargingstations would be necessary (not including the chargers at home). This can not be implemented in the next five years. But assuming that we could built the charging and communication infrastructure in 5 years the cost would be 47.7 billion euros in investment cost and 4.8 billion euros a year in running cost. Over five years this adds up to 71.7 billion euros.
\\ \\
\noindent
On average a vehicle spends over 90 percent of the day parked. Given the infrastructure a EV could be connected to the grid and function as a storage system in this time. In Germany are over 83000 EVs and almost 67000 PHEVs (01 Jan 2019)
%(Stand 2019 %\url{https://www.kba.de/DE/Statistik/Fahrzeuge/Bestand/b_jahresbilanz.html})
and this number is growing exponentially. The Government has the goal to increase this number to 1 million by 2022.
A study by the Frauenhofer institute from 2010 showed with simulations, that a Vehicle to Grid System could provide up to 3.5 kWh of capacity and 0.875 kW of regulation power per Vehicle. This study is now almost 10 years old and the capacities for batteries in EVs have increased a lot since then. But this study uses a very complex simulation which does not just use averages but accounts for different driving behavior at weekends, battery degeneration, dispatch time, different charges at day and night, and a whole lot more. Because of this its results are still viable today but it should be clear that the numbers will increase with improved batteries.
%Seite 31: https://www.isi.fraunhofer.de/content/dam/isi/dokumente/sustainability-innovation/2010/WP4-2010_V2G-Valuation.pdf
\\ \\
This would mean that today the system would have a theoretical capacity of 525 MWh and a regulation power of 131.25 MW. These numbers a relatively low, but the number of EVs and PHEVs in Germany is growing. With 1 million vehicles in the system it would have a theoretical capacity of 3.5 GWh and a regulation power of 875 MW. Assuming that 90 percent of germanys vehicles (42 million vehicles) would be EVs or PHEVs it would result in a theoretical capacity of 147 GWh and a regulation power of 36.75 GW.
\\ \\
\noindent
In order to operate such a system, additional infrastructure is needed. Wherever the EV is parked it needs a connection to the grid via a chargingstation. This means that additional to the fast charging gird on the highway a lot more charchingstations in the cities, at work and anywhere a car might get parked are needed. These chargingstations also need to communicate with the gird in order to make the regulation and storage system work.
At the moment there are 17500 charchingstations in Germany, but 83000 EVs and 67000 PHEVs (01 Jan 2019).
%\url{https://www.kba.de/DE/Statistik/Fahrzeuge/Bestand/b_jahresbilanz.html}
%\url{https://de.statista.com/statistik/daten/studie/460234/umfrage/ladestationen-fuer-elektroautos-in-deutschland-monatlich/}
\\ \\
With the help of numbers provided by Volkswagen we can calculate the costs. Assuming that every owner of an EV or and PHEV already has a charchingstaion at home we only need to install additional ones at workplaces, car parks and public places. But the ones at home still need a connection for the load management. Using the example given by Volkswagen a charchingstation, which provides place for 22 vehicles would have an investment of about 105000 Euro and 250 Euro upkeep every month. The connection for the charger at home would cost about 350 Euro a year.
%\url{https://www.volkswagenag.com/presence/konzern/group-fleet/dokumente/Compendium_Electric_charging_for_fleets_DE.pdf}
\\ \\
To ensure that the EVs and PHEVs can connect almost everywhere they park we would need about 9100 charchingstations from the example.
The cost would then add up to about 960 million Euros of investment cost and 27.5 million per year to run them. And an additionally 52.5 million per year to run the charginstations at home.
%\url{https://www.kba.de/DE/Statistik/Fahrzeuge/Bestand/b_jahresbilanz.html}
If we assume 1 million vehicles in the system the cost would add up to 6.4 billion in investment cost and 532 million per year to run all the charchingstations.
\\ \\
\noindent
The time needed to build all this new chargingstations is comparable to the construction of the Tesla Superchargers. Since 2012 Tesla built almost 15000 individual superchargers at 1650 locations and an additional 24000 destination chargers at hotels worldwide. This would mean for less than 40000 charchingstations it took almost eight years.
%\url{https://en.wikipedia.org/wiki/Tesla_Supercharger}
The superchargers have a power output higher than the ones needed for Vehicle to Grid. Furthermore, the stations are created worldwide. A Charchingsolution in Germany with less powerful chargers would be quicker to realize. The numbers we used form the volkswagenag suggest a time of less than 5 months from planing to finishing the construction of one of the charchingstions from our example. When we keep all this in mind it becomes clear, that it would take approximatly ten years to built all the chargingstions needed for the EVs and PHEVs today. This does not take into account, that the number of EVs and PHEVs is rising exponentially.
\\ \\
\noindent
Unidirectional Vehicle to Grid does not require multiple conversions form AC to DC and vice versa like bidirectional Vehicle to Grid would. The efficiency is comparable to the normal charching efficiency which is on average 65 – 75 percent. This loss efficiency is explained by the different design criteria of the converters. When the charger and the cars converter design match each other, efficiency can be as high as 90 percent. Since the EVs and PHEVs constantly use their charge to drive it is not necessary to include the efficiency losses by holding the charge.%\url{https://backend.orbit.dtu.dk/ws/portalfiles/portal/137328554/efficiency_paper.pdf}
\\ \\
\noindent
The environmental effects of Vehicle to Grid are hard to calculate, since it mostly relies on hardware, that already exist. With unidirectional Vehicle to Gird the additional wear on the battery is negligible. There are no numbers to be found how much Co2 and water the construction of a chargingstation consumes. But with the high amount of chargers needed it should not be ignored.
\\ \\
\noindent
The safety of this system is comparable to the safety of an EV or PHEV charging on a normal chargingstaion. This already controlled and regulated by German law and can therefore be regard as safe.%https://www.bmwi.de/Redaktion/DE/Downloads/V/verordnung-ladeeinrichtungen-elektromobile-kabinettbeschluss.pdf?__blob=publicationFile&v=3
\\ \\
\noindent
Vehicle to Grid is not viable for a system this large. It may fulfill the chosen criteria in Efficiency and Safety but it is too expensive and requires too much new infrastructure to function properly. And even if it would be cost efficient it would not be reasonable to build that many chargingstations. Furthermore, it is not even possible to build that many chargingstations in the given amount of time.
\\ \\
\noindent
Based on the analysis it is recommended, that Vehicle to Grid should not be implemented. It is too expensive and requires too much new infrastructure to function properly. On a small scale there may be cases where Vehicle to grid makes sense. For example a small city, that already has lot of EVs or PHEVs could use a Vehicle to grid system to for its local wind or solar power plants. Or an owner of an owner of a small solar power plant on the rove of its house could use his EV as a storage system for its own power production. Given these requirements a Vehicle to Grid system should be considered. But it is not good solution to utilize the unused renewable energy in Germany.
\\ \\
\noindent
At first glanz Vehicle to Grid seems like a brilliant idea. But as this analysis showed it is neither economical nor technical feasible. It also comes with a social problem, that has not been described in the analysis yet. All calculations assume that the owner of an EV or PHEV has no problem with a software deciding when to charge his car. This could come with problems such as People having an important meeting and not having enough charge, just because the software assumed that the driver wasn't going to work this early. The drivers of PHEVs could end up driving with gasoline most of the time, because the software can completely discharge them. Other than EVs PHEVs can drive with an empty battery, but this does not mean, that the driver is willing to drive with an empty battery.
}

\colchunk{\begin{tiny}
\\\\ \url{https://en.wikipedia.org/wiki/Vehicle-to-grid}\\ \\ \\ \\ \\ \\ \\ \\ \\ \\  \url{https://www.isi.fraunhofer.de/content/dam/isi/dokumente/sustainability-innovation/2010/WP4-2010_V2G-Valuation.pdf}\\ \\ \url{https://www.erneuerbar-mobil.de/sites/default/files/publications/anhang-optum-ap6_1.pdf}\\ \\ \\ \\ \\ \\ \\ \\ \\ \\ \url{https://www.bdew.de/presse/presseinformationen/zahl-der-woche-gut-32-mrd-kilowattstunden/}\\ \\ \\ \\ \\ \\ \\ \\ \\ \\ \\ \\ \\ \\ \\ \\ \\ \\ \\ \\ \\ \\ \\ \\ \\ \\ \\ \\ \\ \\ \\ \\ \url{https://www.kba.de/DE/Statistik/Fahrzeuge/Bestand/b_jahresbilanz.html} \\ \\ \\ \\ \url{https://www.isi.fraunhofer.de/content/dam/isi/dokumente/sustainability-innovation/2010/WP4-2010_V2G-Valuation.pdf}
\\\\\\\\\\\\\\\\\\\\\\\\\\\\\\\\\\\\\\\\\\\\\\\url{https://de.statista.com/statistik/daten/studie/460234/umfrage/ladestationen-fuer-elektroautos-in-deutschland-monatlich/}\\ \\ \url{https://www.kba.de/DE/Statistik/Fahrzeuge/Bestand/b_jahresbilanz.html} \\ \\ \\ \\ \\ \url{https://www.volkswagenag.com/presence/konzern/group-fleet/dokumente/Compendium_Electric_charging_for_fleets_DE.pdf} \\ \\ \\ \\ \\ \\ \\ \\ \\\\\\\\\\\\\\\\\\ \url{https://en.wikipedia.org/wiki/Tesla_Supercharger}\\ \\ \\ \\ \\ \\ \\ \\ \\ \\ \\ \\ \\ \\\\\\\\\\\\\\ \url{https://backend.orbit.dtu.dk/ws/portalfiles/portal/137328554/efficiency_paper.pdf}\\ \\ \\ \\ \\ \\ \\ \\ \\ \\ \\ \\ \\\url{https://www.bmwi.de/Redaktion/DE/Downloads/V/verordnung-ladeeinrichtungen-elektromobile-kabinettbeschluss.pdf?__blob=publicationFile&v=3}\\ 
\url{}
\end{tiny}
}
\end{parcolumns}
\begin{flushright}
Revised: 28 Jan 2020
\end{flushright}
\clearpage
\cfoot{}