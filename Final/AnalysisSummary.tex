\begin{parcolumns}[colwidths={1=2.5 cm, 2=10 cm, 3=2.5cm}]{3}

\textbf{Conclusions} \\ Team Conclusions \\ \\ \textbf{Findings} \\ \\ \textbf{Criteria Discussion} \\ \\ \textbf{Suggestions & Recommendations} 

\colchunk{\\ \\
Based on the discussion above Hydrogen storage technology as a closed system can not be economical feasible. Even if the cost can be reduced by using a combined cycle it is still significantly higher than the market price. Nevertheless, Hydrogen can be made safely from renewable energy sources, is fully scalible and is virtually non-polluting. It will definitely join electricity as an important energy carrier in the future but not as an energy storage medium. We do not recommend the implementation of the system. 
\noident
Based on the analysis it is recommended, that Vehicle to Grid should not be implemented. It is to expensive and requires to much new infrastructure to function properly. On a small scale there may be cases where Vehicle to grid makes sense. For example a small city, that already has lot of EVs or PHEVs could use a Vehicle to grid system to for its local wind or solar power plants. Or an owner of an owner of a small solar power plant on the rove of its house could use his EV as a storage system for its own power production. Given these requirements a Vehicle to Grid system should be considered. But it is not good solution to utilize the unused renewable energy in Germany.
\noident
All of these technologies may become feasible in the near future. At the moment, however, a conventional PSH solution is not feasible in Germany as the necessary terrain to accommodate a storage system of the required size does not exist. Germany is already serviced by its neighbours Luxembourg, Switzerland and Austria to store electrical energy in PSH systems. If Norwegian PSH system can also be used or expanded, a solution using PSH is feasible.
\noident
If the technology would be used right now, it would not be reasonable. Power-to-Gas in its current state is uneconomic, because of high investment costs, electricity prices and expensive electrolyzers. Despite this, it may become a viable method for any electrical power input in the future, since it is easy scalable for a given amount of excess power and hydrogen has diverse uses. Scientific researches show that commercialization and lower prices for power can make Power-to-Gas a reasonable application in the next 15 to 20 years.
\noident


\colchunk{\begin{tiny}
references
\end{tiny}
}
\end{parcolumns}
\begin{flushright}
Revised: 20 JAN 2020
\end{flushright}
\clearpage