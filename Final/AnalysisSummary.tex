\begin{parcolumns}[colwidths={1=2.5 cm, 2=10 cm, 3=2.5cm}]{3}
\colchunk{ \\ \noindent \textbf{Topic and Contributions} \\ \\ \\ \\ \\ \\ \\ \\ \\ \\ \\ \\ \\ \textbf{Findings} \\ Vehicle-2-Grid \\ \\ \\ \\ \\ \\ Pumped Storage Technology \\ \\ \\ \\ \\ \\ \\ \\ \\ \\ \\ \\ \\ Hydrogen Storage Technology \\ \\ \\ \\ \\ \\ \\  Power-to-Gas \\ \\ \\ \\ \\ \\ \\ \\ \\ \\ \\ Battery Storage Technology
}
\colchunk{\\ \noindent Due to climate change and air pollution, an energy revolution is needed. The energy market has to transform into a sustainable yet safe and economically feasible future. To achieve grid security, while having a high percentage of renewables that fluctuate in energy production, energy storage are of utmost importance. This report focused on five of these solutions.
\noindent
\\ 
\begin{itemize}
\item Vehicle-2-Grid
\item Pumped Storage Hydropower
\item Hydrogen storage technology
\item Power-to-Gas
\item Battery Storage Technology
\end{itemize}
\noindent \\ \\
Vehicle-2-Grid offers great resource efficiency but is heavily limited by the quantity of cars connected to the grid. It is too expensive and requires too much new charging infrastructure to function properly. However, it is a safe and efficient technology which could work on a small scale. For our use-case it fails the requirements. \\
\noindent
Pumped Storage Hydropower is a mature and safe storage system but is limited by geographical restraints. For the required power and capacity a conventional Pumped Storage Hydropower System meets the cost requirement, assuming optimal to good system prerequisites. Additionally PSH stores energy efficiently and is easily scalable for high power and capacity systems. However, a conventional PSH solution is not technically feasible in Germany as the necessary terrain to accommodate a storage system of the required size does not exist. Germany is already serviced by its neighbours Luxembourg, Switzerland and Austria to store electrical energy in PSH systems. If Norwegian PSH systems can also be used or expanded, a solution using PSH is feasible. \\  
\noindent
Hydrogen storage technology as a closed system can not be economical feasible. Even if the cost can be reduced by using a combined cycle it is still significantly higher than the market price. Nevertheless, Hydrogen can be made safely from renewable energy sources, is fully scalable and is virtually non-polluting. It will definitely join electricity as an important energy carrier in the future but not as an energy storage medium. \\
\noindent
Power-to-Gas in its current state is uneconomic, because of high investment costs, electricity prices and expensive electrolyzers. Despite this, it may become a viable method for any electrical power input in the future, since it is easy scalable and safe for a given amount of excess power. Additionally Power-to-Gas is efficient and has diverse uses. With the right infrastructure it could be used as fuel or directly fed into the gas grid. Scientific researches show that commercialization and lower prices for power can make Power-to-Gas a feasible application in the next 15 to 20 years. \\
\noindent
Finally Battery Storage Technology offers great adaptability to any given change in requirements, is cost-effective however fails in respects to sustainability. Currently vanadium redox flow batteries are the most suitable solution to the problem. They are easily scalable and provide an expected round-trip efficiency of 75 $\%$. Because of its early stage of development problems in construction could occur. The best implementation is one big storage facility near a wind park which has enough space for later expansions. \\ 
}

\colchunk{\\ \begin{tiny}
\\ \\ \\ \\ \\ \\ \\ \\ 
Kai \\
Lennart \\ 
Moritz \\ 
Christian \\ 
Annabelle \\
\\ \\
see Page \pageref{V2G_Kai} \\ \\ \\ \\ \\ \\
see Page \pageref{PumpedStorageHydropower_Lennart} \\ \\ \\ \\ \\ \\ \\ \\ \\ \\ \\ \\ \\ \\
see Page \pageref{H2_Moritz} \\ \\ \\ \\ \\ \\ \\ \\
see Page \pageref{PowertoGas_Christian} \\ \\ \\ \\ \\ \\ \\ \\ \\ \\ \\
see Page \pageref{Batteries} 
\end{tiny}
}
\end{parcolumns}

\begin{table}[h]
\centering 
\begin{tabular}[h]{cccccc}
Criteria / Systems & Cost & Efficiency & Safety & Scalability & Technical Feasibility \\ 
V2G & -3 & 1 & 3 & -2 & -3 \\
PSH & 1 & 3 & 2 & 2 & -2 \\
HST & -2 & -3 & 2 & 3 & 3 \\
P2G & -2 & 1 & 2 & 3 & 3 \\
Battery & 3 & 2 & 2 & 3 & 2 \\
\end{tabular}
\end{table}
\centerline{Good		+3	+2	+1	-1	-2	-3		  Bad}
\begin{parcolumns}[colwidths={1=2.5 cm, 2=10 cm, 3=2.5cm}]{3}
\colchunk{\\ \noindent \textbf{Criteria Discussion} \\ \\ \\ \\ \\ \\ \\ \\ \\ \\ \\ \\ \\ \\ \textbf{Recommend Solutions} }
\colchunk{\\ \noindent The given framework to rate each system is weighted as follows: 
\begin{itemize}
\item Cost [3x]
\item Efficiency [1x]
\item Safety [1x]
\item Scalability [2x]
\item Technical Feasibility [3x]
\end{itemize}
This results in -20 for Vehicle-2-Grid, +6 for Pumped Storage Hydropower, +8 for Hydrogen Storage Technology, +12 for Power-to-Gas and +25 for Battery Storge Technology. \\
\noindent
During the process it was noticed that the infrastructure of each energy carrier such as Hydrogen or Electricity is additionally an important factor that should be taken into consideration. \\
\noindent
It is recommend to start the construction of a Battery Storage Facility as soon as possible. The facility should be planned and constructed in stages to incorporate further development in Battery Technology. \\
\noindent 
Rapid development in the infrastructure and usage of Hydrogen could alter the outcome. It is recommend to commission another analysis report if such a development signalizes. 
}

\colchunk{\\ \begin{tiny}
references
\end{tiny}
}
\end{parcolumns}
\begin{flushright}
Revised: 20 JAN 2020
\end{flushright}
\clearpage
