\subsection{Hydrogen as an Energy Storage System}
\cfoot{Moritz Gimpel-Henning}
\begin{parcolumns}[colwidths={1=2.5 cm, 2=10 cm, 3=2.5cm}]{3}

\colchunk{  \\ \textbf{Introduction} \\ \\ \\ \\ \\ \\ \\ \\ \\ \\ \\ \\ \\ \\ \\ \\ \\ \\ \\ \\ \\ \\ \\ \\ \\ \\ \\ \\ \textbf{Findings} \\ \\ \\ \\ \\ \\ \\ \\ \\ \\ \\ \\ \\ \\ \\ \\ \\ \\ \\ \\ \\ Electrolysis \\ \\ \\ \\ \\ \\ \\ \\ \\ \\ \\ \\ \\ \\ \\ Storage Technologies \\ \\ \\ \\ \\ \\ \\ \\ \\ \\ \\ \\ The Fuel Cell \\ \\ \\ \\ \\ \\ \\ \\ \\ \\ \\ \\ \\ \\ \\ \\ \\ \textbf{Solution Discussion} \\ \\ \\ \\ \\ \\ \\ \\ \\ \\ \\ \\ \\ \\ \\ \\ \\ \\ \\ \\ \\ \\ \\ \\ \\ \\ \\ \\ \\ \\ \\ \\ \\ \\ \\ \\ \\ \\ \\ \\ \\ \\ \\ \\ \\ \\ \\ \\ \\ \\ \\ \\ \\ Cost \\ \\ \\ \\ \\ \\ \\ \\ \\\ \\ \\ \\ \\ \\ \\ \\ \\ \\ \\ \\ \\ \\ \\ Efficiency \\ \\ \\ \\ \\ \\ \\ \\ \\ \\ \\ \\ \\ \\ \\ Scaling \\ \\ \\ \\ Safety \\ \\ \\ Technical Feasibility \\ \\ \\ \textbf{Conclusion} 
				
			 	}

\colchunk{\\ The share of renewables in the energy-mix continues to grow as more countries pledge themselves to achieve net-zero carbon-output in their power production in the next decades. This seemingly free power manifests with no possibility to regulate its output to the demand of the market. This is especially true for wind energy from off-shore platforms. Windy days can be followed by days with little to no wind. A system with 100\% renewables offers therefore no grid security without methods to overcome the immense mismatch in supply and demand. A feasible solution to that problem is to store excess energy and release it again when needed. One medium which gained a lot of momentum in last years is Hydrogen. Hydrogen energy storage is another form of chemical energy storage in which electrical power is converted into hydrogen. This energy can then be released again by using fuel cells. Hydrogen can be produced from electricity by the electrolysis of water, a simple process. The hydrogen must then be stored, potentially in underground caverns for large-scale energy storage, although steel containers can be used for smaller scale storage. \\
This report presents the basics of hydrogen energy storing systems, a framework to rate and compare this system to other forms of storing energy and finally quantifying the system itself. This discussion underpins the preliminary recommendation that completes the report.
\\ \\
\noindent
To understand the methods to rate the system, the framework itself, we first need to understand how hydrogen is produced, stored and then converted back to electricity. Hydrogen can be considered as the simplest element in existence. Hydrogen is also one of the most abundant elements in the earth’s crust. However, hydrogen as a gas is not found naturally on Earth and must be manufactured. This is because hydrogen gas is lighter than air and rises into the atmosphere as a result. Natural hydrogen is always associated with other elements in compound form such as water, coal and petroleum. Hydrogen has the highest energy content of any common fuel by weight. On the other hand, hydrogen has the lowest energy content by volume. It is the lightest element, and it is a gas at normal temperature and pressure. Hydrogen is considered as a secondary source of energy, commonly referred to as an energy carrier. Energy carriers are used to move, store and deliver energy in a form that can be easily used. Electricity is the most well-known example of an energy carrier. 
\\ \\ 
\noindent
Since hydrogen does not exist on Earth as a gas, it must be separated from other compounds. The most common way for the production of hydrogen is electrolysis. Electrolysis involves passing an electric current through water to separate water into its basic elements, hydrogen and oxygen. Hydrogen is then collected at the negatively charged cathode and oxygen at the positive anode. Hydrogen produced by electrolysis is extremely pure, and results in no emissions since electricity from renewable energy sources can be used. Unfortunately, electrolysis is currently a very expensive process, but costs may fall if the cost of electricity to carry out the procedure also falls. There are also several experimental methods of producing hydrogen such as photo-electrolysis and biomass gasification. \\
\noindent
Hydrogen, which has been created like this must then be stored. Hydrogen as an energy storage medium of the future has this specific advantage. It can be stored in large volume in a number of different ways, including compressed hydrogen in tanks, through chemical compounds that release hydrogen after heating or underground hydrogen storage. The last of which offering the cheapest solution. Underground caverns, salt domes and depleted oil and gas fields can provide the needed space. This possibility is especially cost-effective because almost no construction work is required and also the maintenance practically goes to zero. This method is of course locally restricted. \\
\noindent
Fuel cells directly convert the chemical energy in hydrogen to electricity, with pure water and heat as the only by-products. Its main components are an anode and a cathode which are separated by an electrolyte and a catalyst like platinum. Hydrogen which is pressure fed into the fuel cell on the anode-side first hits the catalyst. It splits H2 into two ions and two electrons. The electrolyte is a proton exchange membrane so it only lets the ions pass. The electrons are picked up by the anode and travel through a cable to the cathode. This electrical current is used to power the grid. On the cathode-side the electrons and ions are reunited and combined with O2 to create water. Hydrogen-powered are not only pollution-free, but a two- to three-fold increase in the efficiency can be experienced when compared to traditional combustion technologies. \\ \\
\noindent
Energy storage technologies are generally compared in terms of their cost, efficiency, scalability, safety and technical feasibility. Hydrogen Energy Storage are also often compared based on application-specific benefits and specific characteristics of interest; however, such comparisons did not take into consideration their financial competitiveness. Financial competitiveness of energy storage system is to define the price of stored energy per kWh over 10 years of the energy storage system. We chose 10 years as most machines had to be replaced by then. The cost is compared to the approximated money spent on imported electricity. The system has of course be as cheap or cheaper than buying the electricity. In all other cases financial feasibility can not be achieved. The Electric Power Research Institute has developed and documented a method that analyses the costs associated with grid connected energy storage applications. It includes the capital and operational expenditures including the upfront capital costs, the fuel expenses, the operating and maintenance charges, the financing costs, etc. Levelized costs can be done using limited input data, thus useful for evaluating technologies with limited operating experience or available data. The Hydrogen energy storage system has not been included in the Electric Power Research Institute analysis or in other cost analysis modeling techniques available in literature. Hydrogen energy storage system appears to be commonly excluded from the cost comparative studies due to its high capital cost and low turnaround efficiency compared to other bulk energy storage systems.  Therefore the cost are calculated in the next chapter. 100 \% of the H2 gas stored is as electricity injected back to the power grid through the FC electricity generation. The by-product O2 is sold. Efficiency was chosen even if it is somewhat represented in costs because it also gives context for sustainability and scalability. Systems that achieve 70\% round-trip efficiency are in the following considered as efficient. Scalability has be taken in account to review the field of usage for this technology. The model was inspired by the energy loss on Off-Shore platforms in the northern part of Germany. The energy loss was determined to be 3.23 TWh. The median for the power loss therefore comes to down to 1.5 GW. To able to save one day worth of energy to achieve grid safety, the system needs to be able to save 36 GWh of energy. Peaks will not be accounted for otherwise cost-effectiveness could also not be achieved. For any energy storage technology safety has be taken into account. It factors in into possible human causalities, loss of revenue stream and grid security. Finally the system has to be able to be integrated into the grid in the next five years. This narrow time window is chosen for this technology would have to be integrated first to enable future rapid expansion of rewenables.\\
\noindent
Typically, the capital costs of an energy storage facility are expressed as \euro /kW installed, where it includes all expenses involved in the purchase and installation of facility. The \euro /kW capital expenditure multiplied by the size of the facility produces the total cost of the project. In the proposed analysis, all the costs related to an energy storage facility are expressed as total \euro /kW of usable discharge capacity (in kW) and total \euro /kWh of usable energy storage capacity. Energy storage technology with deeper Depth of Discharge and higher turn-around efficiency will have a lower unit cost of usable power and energy. The electrolyzer costs 2500 \euro per kW and the fuel cell 4000 \euro for each kWh needed as capital cost. The storage costs are minimal because of the usage of underground hydrogen storage. The costs are factored in with 3 \euro per kWh.  The Operation costs are as follows 50 \euro per Kw for the electrolyzer, 100 \euro per kWh for the fuel cell and 0.2 \euro per kWh for the storage. The sold 02 was priced with 3000 \euro per ton.  The levelised cost for one kWh is therefore 22 cents. Comparing that to the price which was paid for importing the electricity the cost is almost four-times as high. (0.0528/kWh to 0.22/kWh) \\
\noindent
The round-trip efficiency measures up to 45 \% which explains the high cost. The system is therefore not efficient. The efficiency of the electrolysis and the fuel cells is high, but by combining many processes the efficiency quickly goes down.  Their efficiency can be increased by utilizing the output heat from electrolyzers and fuel cells in process heating. Additionally they do not only store the electrical energy for future re-use like all other conventional energy storage systems, but also allow both hydrogen and oxygen gases to be sold as commodities thus increasing the system economic efficiency. The 3:1 increase in revenue options, opens the potential for downstream applications like car fuelling, fertilizer production, and high and low-grade heat applications in addition to electricity. \\
\noindent
Hydrogen storage technology is fully scalible. The power-output is not depended on the energy storage capacity. Our model is therefore achieveable with hydrogen, thus fulfilling the criteria. \\
\noindent
Also safety is fully achieved with less than one in a million fuel cells fail. Even less failures happen in the process of electrolysis or storage. \\
\noindent
Hydrogen is technical feasible. With the technology being fully matured a system with such requirements can be implemented in two to three years. \\ \\
\noindent
Based on the discussion above Hydrogen storage technology as a closed system can not be economical feasible. Even if the cost can be reduced by using a combined cycle it is still significantly higher than the market price. Nevertheless, Hydrogen can be made safely from renewable energy sources, is fully scalible and is virtually non-polluting. It will definitely join electricity as an important energy carrier in the future but not as an energy storage medium. We do not recommend the implementation of the system. 
}

\colchunk{\begin{tiny}\\ \url{https://www.researchgate.net/publication/336527927_Hydrogen_Energy_Storage} \\ \\ \\  \\ \\ \\ \\ \\ \\ \\ \url{https://en.wikipedia.org/wiki/Grid_energy_storage#Hydrogen} \\ \\ \\ \\ \\ \\ \\ \\ \\ \\ \\ \\ \\ \\ \\ \\ \\ \url{https://www.irena.org/-/media/Files/IRENA/Agency/Publication/2018/Sep/IRENA_Hydrogen_from_renewable_power_2018.pdf}  \\ \\ \\ \\ \\ \\ \\ \\ \\ \\ \\ \\ \\ \\ \url{https://www.azocleantech.com/article.aspx?ArticleID=29} \\ \\ \\ \\ \\ \\ \\ \\ \\ \\ \\ \\ \\ \url{https://en.wikipedia.org/wiki/Grid_energy_storage#Hydrogen} \\ \\ \\ \\ \\ \\ \\ \\ \\ \\ \\ \\ \\ \url{https://www.hydrogenics.com/technology-resources/hydrogen-technology/fuel-cells/}  \\ \\ \\ \\ \\ \\ \\ \\ \\ \\ \\ \\ \\ \\ \\ \\ \\ \url{https://www.irena.org/-/media/Files/IRENA/Agency/Publication/2018/Sep/IRENA_Hydrogen_from_renewable_power_2018.pdf}  \\ \\ \\ \\ \\ \\ \\ \\ \\ \\ \\ \\ \\ \\ \\  \url{https://www.researchgate.net/publication/336527927_Hydrogen_Energy_Storage}  \\ \\ \\ \\ \\ \\ \\ \\ \\ \\ \\ \\ \\ \\ \\ \\ \\   \url{https://www.nrel.gov/docs/fy10osti/48360.pdf} \\ \\ \\ \\ \\ \\\ \\ \\ \\ \\ \\ \\ \\ \\ \\ \\  \url{https://www.researchgate.net/publication/336527927_Hydrogen_Energy_Storage} \\ \\ \\ \\ \\ \\ \\ \\ \\ \\ \\ \\ \\ \\ \\ \\ \\ \\ \\https://bit.ly/364Wv1p \\ \\ \\ \\ \\ \url{https://en.wikipedia.org/wiki/Combined_cycle_power_plant} \\ \\ \\ \\ \\ \\ \\ \\ \\ \\ \\ \\ \\ \url{https://www.nrel.gov/docs/fy10osti/48360.pdf}  \\\\\\ \end{tiny} }

\end{parcolumns}
\begin{flushright}
Revised: 16 Jan 2020
\end{flushright}
\clearpage
\cfoot{}
