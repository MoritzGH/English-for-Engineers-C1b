\begin{parcolumns}[colwidths={1=2.5 cm, 2=10 cm, 3=2.5cm}]{3}

\colchunk{  \\ \textbf{Study Project} \\ \\ \\ \\ \\ \textbf{Work Done} \\ \\ \\ \\ \\ \\ Criteria \\ \\ \\ \\ \\ \\ \\ \\ \\ \\ \\ \\ \\ \\ \\ \\ \\ \\ \\ \\ \\ \textbf{Problems encountered} \\ \\ \\ \\ \\  Cost \\ \\ \\ Efficiency \\ \\ Safety \\ \\ \\ \\ Scalability \\ \\ \\ \\ \\ Technical Feasibility \\ \\ \\ \\ \\ \\  \textbf{Recommend Solutions} \\ \\ \\ \\ \\ \\ \\ \\ \textbf{Total Cost}  
				
				}

\colchunk{\\ Five different approaches to store excess electricity from wind mills have been analyzed by \textbf{Rocket Science}.
\\ This report recommends what technologies are to be used in the future. 
\\ \\
The evaluated Energy Storing Technologies were:
 \begin{itemize}
\item V2G (Vehicle-2-Grid)
\item PSH (Pumped Storage Hydropower)
\item HST (Hydrogen storage technology)
\item P2G (Power-to-Gas)
\item Batteries (Battery Storage Technology)
\end{itemize}
To determine the feasibility of the each technology, they were compared to a framework in which the needed electricity is bought. The following Criteria were used:
 \begin{itemize}
\item Cost 
\item Efficiency
\item Safety
\item Scalability
\item Technical Feasibility
\end{itemize}
\noindent Cost and Technical Feasibility were weighted as exclusive criterion. Generally speaking, the cheapest option prevails in the market. But that doesn't matter if it can't be built safely and fast. Scalability was rated as an important factor to enable future expansion and a wider usage.  \\
Efficiency was rated normally because the cost perspective is already included in Cost. But Efficiency also incorporates resource usage and therefore Sustainability which was considered important to cover. Safety was included to ensure the Safety of personnel and revenue stream.\\


\noindent The main problems for each technology were:
 \begin{itemize}
\item limited capacity, not technical feasible
\item geographical restraints, not enough mountains
\item costly, inefficient
\item costly, gas infrastructure
\item harmful chemicals, early stage in development
\end{itemize}

\noindent V2G, HST and P2G are economically not feasible. In a good to best case scenario PSH fulfills the requirements, whereas Batteries fulfill the requirements in all cases.

\noindent All Systems show a round-trip Efficiency of over 70 $\%$ except HSTs which only offer 45 $\%$. 

\noindent V2G is considered the safest out of all technologies because it is heavily decentralized and therefore only handles a fraction of each power output. All systems fail in less than 1 in a million cases. 

\noindent The capacity and power output of P2G, HST and Batteries can be scaled independently to any amount. The capacity of PSH can be scaled up to the available space in the mountains. V2G is heavily restricted by the cars connected to the grid. 

\noindent V2G is not technically feasible. Whereas PSH is technically feasible, but the needed mountains are not available in Germany. If the mountains in for example Norway could be used, PSH would be possible. HST, P2G and Batteries are all technically feasible therefore a system with such requirements could be built in the next five years. \\ \\
\noindent
It is recommend to start the construction of a Battery Storage Facility as soon as possible. The facility should be planned and constructed in stages to incorporate further development in Battery Technology. \\
\noindent 
Rapid development in the infrastructure and usage of Hydrogen could alter the outcome. It is recommend to commission another analysis report if such a development signalizes. \\

\noindent The total cost of the report is 3,678.67 \euro .
}
\colchunk{\begin{tiny} \\ \\ \\ \\ \\ \\ \\
see Page \pageref{V2G_Kai} \\ 
see Page \pageref{PumpedStorageHydropower_Lennart} \\ 
see Page \pageref{H2_Moritz} \\ 
see Page \pageref{PowertoGas_Christian} \\ 
see Page \pageref{Batteries} \\ \\ \\ \\ \\ \\ \\ \\ \\ \\ \\ \\ \\ \\ \\ \\ \\ \\ \\ \\ \\ \\
see Page \pageref{Analysis} \\ see \textbf{V2G} \\ see \textbf{PSH} \\ see \textbf{HST} \\ see\textbf{P2G} \\ see \textbf{Batteries}
\end{tiny}
}
\end{parcolumns}
\begin{flushright}
Revised: 27 JAN 2020
\end{flushright}