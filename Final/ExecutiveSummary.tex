\begin{parcolumns}[colwidths={1=2.5 cm, 2=10 cm, 3=2.5cm}]{3}

\colchunk{  \\ \\ \textbf{Study Project} \\ \\ \\ \textbf{Work Done} \\ \\ \\ \textbf{Problems encountered} \\ \\ \\ \textbf{Analysis of Problems} \\ \\ \\ \textbf{Recommend Solutions} \\ \\ \\ \textbf{Cost} 
				
				}

\colchunk{\\ Five different approaches to store excess electricity from wind mills have been analyzed by \textbf{Rocket Science}.
\\ This report shows what technologies are to be used in the future. 
\\
The evaluated Energy Storing Technologies were:
 \begin{itemize}
\item Vehicle-2-Grid
\item Pumped Storage Hydropower
\item Hydrogen storage technology
\item Power-to-Gas
\item Battery Storage Technology
\end{itemize}
To determine the feasibility of the each technology, they were compared to a framework in which the needed electricity is bought. The following Criteria were used:
 \begin{itemize}
\item Cost 
\item Efficiency
\item Safety
\item Scalability
\item Technical Feasibility
\end{itemize}
\\ \noindent Cost and Technical Feasibility were weighted as exclusive criterion. Generally speaking, the cheapest option prevails in the market. But that doesn't matter if it can't be built safely and fast. Scalibility was rated as an important factor to enable future expansion. 


\colchunk{\\\,\,\,\,\, \textbf{References}
\\\\\\ }

\end{parcolumns}