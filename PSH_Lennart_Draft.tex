\documentclass[12pt,a4paper]{article}
\renewcommand{\baselinestretch}{1.5}
\usepackage[utf8]{inputenc} 
\usepackage{float}
\usepackage[english]{babel}
\usepackage{amsmath}
\usepackage{wasysym}  
\usepackage{amssymb}  
\usepackage{amsfonts} 
\usepackage{graphicx} 
\usepackage{hyperref} 
\usepackage{booktabs} 
\usepackage{todonotes} 
\usepackage{marginnote}
\usepackage[official]{eurosym}
\usepackage{lastpage}
\usepackage{parcolumns}


\usepackage[a4paper,bindingoffset=0.25in,
top= 1in,left=0.5in,right=1in,bottom=1in,
footskip=.25in]{geometry}
\setlength{\headsep}{40pt}
\usepackage[headsepline,plainheadsepline]{scrpage2} 
\pagestyle{scrheadings} 

\clearscrheadings 
\clearscrplain 
\clearscrheadfoot 

\ihead[{\includegraphics[height=40pt]{logo}}]{\includegraphics[height=40pt]{logo}}
\ohead{\headmark}
\automark[subsection]{section}
\ofoot{\pagemark}



\begin{document}


\section{Section IV: Pumped Storage Hydropower}
\begin{parcolumns}[colwidths={1=2.5 cm, 2=10 cm, 3=2.5cm}]{3}

\colchunk{  \\ \\ \\ \\ \\ \\ \\ \\ \\ \\ Unused energy \\ \\ \\ \\ \\ \\ \\ \\ Requirements \\ \\ \\ \\ \\ \\ \\ \textbf{Costs} \\ \\ \\ \\ \\ Capital costs \\ \\ \\ \\ \\ \\ \\ \\ \\ \\ \\ \\ \\ \\ \\ \\ \\ \\ \\ \\ \\ Small scale PSH \\ \\ \\ \\ \\ \\ \\ \\ \\ \\ \\ Running costs \\ \\ \\ \\ \\ \\ \\ \\ \\ Total costs \\ \\ \\ \\ \\ \textbf{Efficiency} \\ \\ \\ \\ \\ \\ \\ \\ \\ \\ \\ \\ \\ Total efficiency \\ \\ \\ \\ \\ \textbf{Response time} \\ \\ \\ \\ \\ \\ \\ \\ \\ \\ \textbf{Scalability} \\ \\ \\ \\ \\ \\ \\ \\ \\ \\ \\ \\ \\ \\ \textbf{Sustainability} \\ \\ \\ \\ \\ \\ \\ \\ \\ \\ \\ \\ \\ \\ \\ \\ \\ \\ \\ \\ \textbf{Technical \\ feasibility} \\ \\ \\ \\ \\ \\ \\ \\ \\ \\ \\ Seawater PSH \\ \\ \\ \\ \\ \\ \\ \\ Mine PSH \\ \\ \\ \\ Closed container PSH \\ \\ \\ \\ Underwater PSH \\ \\ \\ \\ \\ \\ \\ \\ \\ Conclusion \\
				
				}

\colchunk{\\ \\ PSH (Pumped Storage Hydropower) is an electrical storage technology that has been used since the 1890s. It works on the principle of changing the gravitational potential energy of water. This is done by first pumping the water from a place of low elevation to a higher one to store energy and then, when electrical energy is required, releasing that water to run through a turbine and generate electricity.

In Quarter 1 of 2019 an estimated 3,230,000,000 kWh were produced by wind turbines in Germany that could not be used or transported to a location where it could have been used. This means that an average of 1,495 MWh are left unused every hour. Germany already has 6,806 MW of existing PSH. However, as this is already used for load balancing it will not be taken into account for still existing unused power.

To calculate an example of the required storage system in Germany the power requirement for such a storage system is set at the average unused power of 1,495 MW. A days worth of this average unused power comes to 35,880 MWh of required stored energy. This value will be used as the required capacity. Additionally this storage system should have a lifetime of at least 10 years.

For conventional PSH the initial capital costs are very high while the operational costs are low. With a typical life cycle of 20 to 50 years and for some locations even up to 75 years, this means that PSH will pay dividends only after a long running time.

Typical upfront capital costs include the acquisition of the land, the machines required to generate electricity and pump water, the structure that regulates the flow of water to the reservoir, and the infrastructure to connect the PSH to the grid.

A german study of 11 such structures found the upfront capital cost to be 1850 USD per kW of power generated. This value takes into account an energy to power ratio (E/P) of between 6 to 20. This means that for every 1 kW of power generation between 6 to 20 kWh of storage capacity is installed. Assuming 1495 MW is left unused at all times, this would result in an upfront capital investment of 2.8 billion USD. A storage capacity of 40 GWh is required which, at 1495 MW, results in an E/P of 26.8. This is above the typical E/P values and thus may result in higher costs.

Capacity capital costs range between 70 to 230 USD per kWh according to Kamath and 250 to 350 USD per kWh according to May. Using these values, a capacity of 35.9 GWh may range from 2.5 billion USD to 12.6 billion USD.

However, these values are only estimates for large PSH at roughly 500 MW. For PSH with 100 MW or lower the capital costs steadily increase. A small scale PSH by Shell Energy North America (SENA) rated at 5 MW and with a capacity of 30 MWh cost a total of 22.3 million USD in capital costs. This translates to a much higher rate of 4,400 USD per kW and 743 USD per kWh. This makes conventional PSH cost inefficient for small scale endeavours. As the technology improves and costs decrease this may be a suitable solution in the near future.

The typical running costs of PSH are very low. Lazard estimates the levelized costs as 152 to 198 USD per MWh. Levelized costs are calculated by dividing the total maintenance costs by the total energy produced over the lifecycle of the PSH. Considering 1,495 MW are lost on average this results in a total of 131 TWh lost over the course of 10 years. If all of this energy is to be stored, this results in 19.9 up to 25.9 billion USD of running costs.

In total the lowest estimate predicts 22.4 billion USD including investment and running costs over a 10 year period. The highest estimate predicts 38.5 billion USD. If small scale PSH are used this value will increase significantly.

PSH lose energy at many points of the storage process. The first losses occur at the entrance to the pump and at the pump. The pump then transports the water over a large distance to a higher elevation. At this point, friction with the pipe walls cause losses. To minimize losses due to friction the pipe must be constructed accordingly by increasing the diameter, reducing the friction coefficient with the pipe wall, reducing the flow rate and reducing the pipe length. Since the last two are important factors to the power generation, the diameter and pipe surface composition can be optimized. If the water is stored in an open container, such as a reservoir lake, evaporation can cause some of the water to be lost.

Considering all the losses, state of the art PSH have an overall efficiency of 80\%. Other sources estimate the round-trip efficiency at 70 to 87\%. PSH is a very mature technology due to its age. This means that much of the efficiency gains have already been put into use.

The response time of a PSH should be considered, as this influences power left unused and therefore the overall efficiency of the PSH as a storage system. A short response time to a surge or shortage of electricity is important, as in the time in which a storage system is not online valuable power will be left unused or demand of electrical power will be left unfulfilled. PSH boasts a time of 60 to 220 seconds from offline to electricity generation and 300 to 360 seconds from offline to pumping.

PSH are easily scalable towards the high end of power generation. As mentioned, costs for low power generation increase drastically to make small conventional PSH less economically feasible. As mentioned in the Costs section, PSH as small as 5 MW can become feasible in the future. 

To create even smaller PSH a different technology is required. The Goudemand building in France has an open water container on its roof connected to a very small scale PSH. However, the concept of PSH does not translate itself well to this size as much of the otherwise high efficiency and economical feasibility is lost. An additional challenge in this urban situation is the access to a very large amount of water.

PSH systems are environmentally sustainable system as they require create virtually no emissions during their operation. The emissions created during construction are also negligible. However, specifically open-loop PSH can create issues with the existing wildlife in the area. A dam can interrupt the downstream flow of sediment in the water. The sediment piles up at the location of the dam the erodes the existing habitat of the species found there. Additionally the existing fish species may not be able to travel upstream to their breeding ground, possibly causing vast populations to perish. To counter this, proper bypasses should be built.

The flow of water in an open-loop PSH along a river should be regulated to avoid sudden large volumes of water travelling downstream. Also, the flow of water should be increased during breeding months of fish to allow for fish to travel upstream. The dependence upon a source of water for these systems is critical. As such a drought or too much rain can have large impacts on the ability of the system to store energy.

Conventional PSH are located in areas with high elevation changes. Typically an elevation difference of 200 to 300m is required. Additionally access to a large amount of water is required to fill the reservoirs. Southern Germany has these required elevation changes and water sources. Subsequently 80\% of all Hydropower is located in Bavaria and Baden-Württemberg. However, this is an issue since the wind energy is mostly produced in northern Germany and has to therefore be transported over long distances. To overcome this geographical restriction of requiring very mountainous terrain, other technologies can be put to use. 

Along the coast seawater PSH can be used. In this situation the upper reservoir is the ocean’s sea level and the lower level is a reservoir far below sea level. Ideally this is done in areas where the land is already below sea level, such as behind dikes. A major issue with this concept the corrosion due to seawater. However, this allows for the PSH to be situated near the coast and near overshore wind turbines.

Old abandoned mines and quarries may be used for PSH. A reservoir on the surface and a second lower reservoir in the mine can be used to create the required gravitational potential.

A PSH with closed containers may be used, where one is filled with compressed air. This requires no elevation change, since the pressure in one container creates the required potential. This PSH is however limited in its size.

An underwater PSH may be used for offshore wind turbines. In this system a container is lowered to the ocean floor. Since the pressure outside of the container is much higher than in the container, pumping water out of the container will require energy, while filling the container with water from outside will generate energy with the use of a turbine. For this technology corrosion is once again a major factor. However, the location so close to the offshore wind farms will prove useful.

All of these technologies may become feasible in the near future. At the moment, however, a conventional PSH solution is not feasible in Germany as the necessary terrain to accommodate a storage system of the required size does not exist. Germany is already serviced by its neighbours Luxembourg, Switzerland and Austria to store electrical energy in PSH systems. If Norwegian PSH system can also be used or expanded, a solution using PSH is feasible.


}

\colchunk{\\\,\,\,\,\, \textbf{References}
\\ \url{https://en.wikipedia.org/wiki/Pumped-storage_hydroelectricity} \\ \\ \\ \\ \\ \\ \url{https://www.bdew.de/} \\ \\ \url{https://www.hydropower.org/country-profiles/germany} \\ \\ \\ \\ \\ \\ \\ \\ \\ \url{https://www.energy.gov/} \\ \url{https://en.wikipedia.org/wiki/Pumped-storage_hydroelectricity} \\ \\ \\ \\ \\ \\ \url{https://www.energy.gov/} \\ \\ \\ \\ \\ \\ \\ \\ \\ \\ Kamath (2016) \\ May (2018) \\ \\ \\ \\ \url{https://www.energy.gov/} \\ \\ \\ \\ \\ \\ \\ \\ \\ \\ \url{https://www.lazard.com/} \\ \\ \\ \\ \\ \\ \\ \\ \\ \\ \\ \\ \\ \\ \\ \\ \\ \url{https://en.wikipedia.org/wiki/Darcy\%E2\%80\%93Weisbach_equation} \\ \\ \\ \\ \\ \\ Yang (2016) \\ \url{https://www.energy.gov/} \\ \\ \\ Yang (2016) \\ \\ \\ \\ \\ \\ \\ \\ \\ \\ \\ \\ \\ \\ \\ \\ \url{https://phys.org/news/2016-10-pumped-storage-hydroelectricity.html} \\ \\ \\ \\ \\ \\ \url{https://www.ifc.org/} \\ \\ \\ \\ \\ \\ \\ \\ \\ \\ \\ \url{https://www.ifc.org/} \\ \\ \\ \\ \\ \\ \\ Yang (2016) \\ \\ \\ \\ \\ \url{https://en.wikipedia.org/wiki/Wind_power_in_Germany} \\ \\ \\ \\ \\ Yang (2016) \\ \\ \\ \\ \\ \\ \\ \\ \\ \\ \\ \\ \\ \\ \\ \\ \\ \\ \\ \\ \\ \\ \\ \\ \\ \\ \url{https://www.hydropower.org/country-profiles/germany} \\ 

}

\end{parcolumns}

\end{document}