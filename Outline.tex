\documentclass[12pt,a4paper]{article}
\usepackage[utf8]{inputenc} 
\usepackage{float}
\usepackage[english]{babel}
\usepackage{amsmath}
\usepackage{wasysym}  
\usepackage{amssymb}  
\usepackage{amsfonts} 
\usepackage{graphicx} 
\usepackage{hyperref} 
\usepackage{booktabs} 
\usepackage{todonotes} 
\usepackage{marginnote}
\usepackage[official]{eurosym}
\usepackage{lastpage}
\usepackage{parcolumns}


\usepackage[a4paper,bindingoffset=0.25in,% 
top= 1in,left=0.5in,right=1in,bottom=1in,%
footskip=.25in]{geometry}
\setlength{\headsep}{40pt}
\usepackage[headsepline,plainheadsepline]{scrpage2} 
\pagestyle{scrheadings} 

\clearscrheadings 
\clearscrplain 
\clearscrheadfoot 

\ihead[{\includegraphics[height=40pt]{logo}}]{\includegraphics[height=40pt]{logo}}
\ohead{\headmark}
\automark[subsection]{section}
\ofoot{\pagemark}



\begin{document}

\section{Outline: Using Hydrogen as Energy Storage}
\begin{parcolumns}[colwidths={1=2.5 cm, 2=10 cm, 3=2.5cm}]{3}

\colchunk{  \\ \\ \textbf{Current Situation} \\ \\ \\ 
				\\ \\ \\	\\ \textbf{Need}\\ \\
				\\ \\ \\ \\ \\	\\ \textbf{Problem}\\ \\ \\
					\textbf{Method and Criteria}
				\\ \\ \\	\textbf{References and Informations}}

\colchunk{\\ \\Each year the electricity generated by using renewable energies like solar and wind makes up a bigger part in the energy-mix of Germany. But the amount varies because of seasonal or just daily fluctuations in wind. In 2018 wind energy was the main renewable source with about 48\%, which  made up around 19\% of the overall consumption.\\ \\
Because of the significant mismatch in grid power demand, the need for an energy storage solution  is becoming more acute. It’s a well-established problem for the industry, and there are a number of energy management and storage systems in the pipeline today, but few offer a complete solution allowing wind energy to be seamlessly plugged into the grid.


}

\colchunk{\\\,\,\,\,\, \textbf{References}}

\end{parcolumns}

\end{document}