\documentclass[12pt,a4paper]{article}
\usepackage[utf8]{inputenc} 
\usepackage{float}
\usepackage[english]{babel}
\usepackage{amsmath}
\usepackage{wasysym}  
\usepackage{amssymb}  
\usepackage{amsfonts} 
\usepackage{graphicx} 
\usepackage{hyperref} 
\usepackage{booktabs} 
\usepackage{todonotes} 
\usepackage{marginnote}
\usepackage[official]{eurosym}
\usepackage{lastpage}
\usepackage{parcolumns}


\usepackage[a4paper,bindingoffset=0.25in,
top= 1in,left=0.5in,right=1in,bottom=1in,
footskip=.25in]{geometry}
\setlength{\headsep}{40pt}
\usepackage[headsepline,plainheadsepline]{scrpage2} 
\pagestyle{scrheadings} 

\clearscrheadings 
\clearscrplain 
\clearscrheadfoot 

\ihead[{\includegraphics[height=40pt]{logo}}]{\includegraphics[height=40pt]{logo}}
\ohead{\headmark}
\automark[subsection]{section}
\ofoot{\pagemark}



\begin{document}

\section{Outline: Different Solutions for Energy Storage}
\begin{parcolumns}[colwidths={1=2.5 cm, 2=10 cm, 3=2.5cm}]{3}

\colchunk{  \\ \\ \textbf{Current Situation} \\ \\ \\ 
				\\ \\ \\	\\ \textbf{Need}\\ \\
				\\ \\ \\ \\ \\	\\ \textbf{Problem}\\ \\ \\
				\\\\\\\\\\\\\\\\\\\\\\\\ \textbf{Method and Criteria}
				\\ \\ \\\\\\\\\\\     \\\\\\ \\\\\\\\\\\\\\\\\\	\\\\ \textbf{References and Informations}}

\colchunk{\\ \\Each year the electricity generated by using renewable energies like solar and wind makes up a bigger part in the energy-mix \protect\footnotemark of Germany. But the amount varies because of seasonal or just daily fluctuations in wind. In 2018 wind energy was the main renewable source with about 48\%, which  made up around 19\% of the overall consumption.\\ \\
Because of the significant mismatch in grid power demand, the need for an energy storage solution  is becoming more acute. It’s a well-established problem for the industry, and there are a number of energy management and storage systems in the pipeline today, but few offer a complete solution allowing wind energy to be seamlessly plugged into the grid.\\\\
Today, the importance of transitioning into a sustainable and cost-effective energy sector is more important than ever. Fossil fuels won’t last for ever and are straining the environment too much. The state will on the long-term ban or at least heavily restrict the usage to meet its own agendas therefore the solution for efficiently storing renewables is of utmost importance.\\
Energy production by wind power is intermittent and fluctuates. Currently one of the main challenges is the adaptability of different energy storage systems to daily and annual fluctuations as well. This paper will investigate whether Smart Grids, Power to Gas, hydrogen cells, pumped-storage stations or batteries are the best solution to those problems.\\ \\
The new energy storage method must be more efficient in the long-term than using large batteries or using pumped hydroelectric energy storage systems. It has to be lower in cost than the current technology and be able to be integrated into the current grid of wind turbines.  The factors to rate hydrogen as an energy storing method therefore include:
\begin{itemize}
\item \textbf{Costs} in \euro: including investment, maintenance and operating costs 
\item \textbf{Efficiency} in kwh: adding kwh lost while transforming and lost while storing over one year
\item \textbf{Safety} in \%: failure rate per year
\item \textbf{Scaling} yes or no: Is it applicable for single wind turbine/cluster of turbines/wind farm? Is it reasonable for an input of X kwh/saving X kwh?
\item \textbf{Technical Feasibility} yes or no: Implementable in the next X years? Is the technology viable or is something better obtainable in the next years? Is it possible in the geographic area? 
\item \textbf{Sustainability} in liters of water consumption and CO$_{2}$ emissions in tons\\
\end{itemize}
[energy-mix]  Der deutsche Strommix: Stromerzeugung in Deutschland. \\https://strom-report.de/strom/.\\Last accessed 21.11.19


}

\colchunk{\\\,\,\,\,\, \textbf{References}
\\\\\\ 1 Source: \url{https://strom-report.de/strom}}

\end{parcolumns}

\end{document}